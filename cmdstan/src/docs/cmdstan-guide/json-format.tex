\chapter{JSON Format}\label{json.chapter}

\noindent 
\CmdStan can use JSON format for input data for both
model data and parameters. Model data is read in by the model
constructor. Model parameters are used to initialize the sampler and
optimizer.

\section{JSON Syntax Summary}

JSON is a data interchange notation, defined by an 
\href{http://www.ecma-international.org/publications/files/ECMA-ST/ECMA-404.pdf}{ECMA standard}.
JSON data files must in Unicode.
JSON data is a series of structural tokens, literal tokens, and values:
\begin{itemize}
\item Structural tokens are the left and right curly bracket, left and right square bracket, the semicolon, and the comma. \{\}[]:,
\item Literal tokens must always be in lowercase. There are three literal tokens: \code{true} \code{false} \code{null}
\item A primitive value is a single token which is either a literal, a string, or a number.
\item A string consists of zero or more Unicode characters enclosed in
  double quotes, e.g. \code{"foo"}.  A backslash is used to escape the double quote
  character as well as the backslash itself. JSON allows the use of
  Unicode character escapes, e.g. \code{\\uHHHH}  where \code{HHHH} is
  the Unicode code point in hex.
\item All numbers are decimal numbers. Scientific notation is
  allowed. The following are examples of numbers  \code{17 17.2 -17.2
    -17.2e8 17.2e-8}  The concepts of positive and negative infinity as well as
  "not a number" cannot be expressed as numbers in JSON, but they can
  be encoded as strings which can be mixed with numbers.
\item A JSON array is an ordered, comma-separated list of zero or more
  JSON values enclosed in square brackets. The elements of an array
  can be of any type.  The following are examples of arrays: \code{[] [1] ["a","b",true]}
\item A name-value pair consists of a string followed by a colon followed by a value, either primitive or compound.
\item A JSON object is a comma-separated series of zero or more
  name-value pairs enclosed in curly brackets. Each name-value pair is
  a member of the object. Membership is unordered. Member names are
  not required to be unique.  The following are examples of objects:
  \code{\{\}  \{"foo": null\} \{"bar" : 17, "baz" : [14,15,16.6] \}}
\end{itemize}

\section{Stan Data Types in JSON Notation}

Stan follows the JSON standard.
A Stan input file in  JSON notation consists of  JSON object which contains zero
or more name-value pairs.  This structure corresponds to a Python data
dictionary object.  The following is an example of JSON data for the
simple Bernoulli example model:
\begin{quote}
\begin{Verbatim}
{ "N" : 10, "y" : [0,1,0,0,0,0,0,0,0,1] }
\end{Verbatim}
\end{quote}
Matrix data and multi-dimensional arrays are indexed in row-major
order.  For a Stan program which has data block
\begin{quote}
\begin{Verbatim}
data {
  int d1;
  int d2;
  int d3;
  int ar[d1, d2, d3];
}
\end{Verbatim}
\end{quote}
\noindent
the following JSON input file would be valid
\begin{quote}
\begin{Verbatim}
{ "d1" : 2,
  "d2" : 3,
  "d3" : 4,
  "ar" : [[[0,1,2,3], [4,5,6,7], [8,9,10,11]],
          [[12,13,14,15], [16,17,18,19], [20,21,22,23]]]
}
\end{Verbatim}
\end{quote}
JSON ignores whitespace. In the above examples,  the spaces and
newlines are only used to improve readability and can be omitted.

All data inputs are encoded as name-value pairs.
The following table provides more examples of JSON data.
The left column contains a Stan data variable declaration
and the right column contains valid JSON data inputs.
%
\begin{center}
\begin{tabular}{r||c|c}
{\it Stan variable} & {\it JSON data} \\ \hline \hline
\code{int i;} & \code{"i" : 17} \\
\\ 
\code{real a;} & \code{"a" : 17} \\
 & \code{"a" : 17.2} \\
 & \code{"a" : "NaN"} \\
 & \code{"a" : "+inf"} \\
 & \code{"a" : "-inf"} \\
\\
\code{int a[5];} & \code{"a" : [1, 2, 3, 4, 5]} \\
\\
\code{real a[5];} & \code{"a" : [ 1, 2, 3.3, "NaN", 5 ]} \\
\code{vector[5] a;} & \code{"a" : [ 1, 2, 3.3, "NaN", 5 ]} \\
\code{row\_vector[5] a;} & \code{"a" : [ 1, 2, 3.3, "NaN", 5 ]} \\
\code{real a[5];} & \code{"a" : [ 1, 2, 3.3, "NaN", 5 ]} \\
\\
\code{matrix[2,3] a;} & \code{"a" : [ [ 1, 2, 3 ], [ 4, 5, 6] ]} \\
\end{tabular}
\end{center}
%


